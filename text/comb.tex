\section{组合容斥}

\subsection{常用公式}

$2n$ 个元素两两组合的方案数为 $\displaystyle\prod_{i=1}^{n}(2i-1)=(2n-1)!!$ 。

卡特兰数:$\displaystyle f(n)={1\over n+1}{2n\choose n}$ 。

\vspace{0.2cm}

错排:$\displaystyle D(n)=(n-1)[D(n-2)+D(n-1)],D(0)=1,D(1)=0$ 。

\vspace{0.2cm}

Lucas定理:$\displaystyle {x\choose y}={x\bmod p\choose y\bmod p}{\lfloor{x\over p}\rfloor\choose \lfloor{y\over p}\rfloor}$ 。

\subsection{组合背包}

$n$ 个物品,$f_{t,i,j}$ 表示前 $t$ 个物品选 $i$ 个,容量为 $j$ 方案数,给定 $K$ ,需要对每个 $j$ 求:
$$
ans_j=\sum_{i=K}^{n}(-1)^{i-K}{i-1\choose K-1}f_{n,i,j}
$$
令 $\displaystyle F_{t,j,k}=\sum_{i=k}^{n}(-1)^{i-k}{i-1\choose k-1}f_{t,i,j}$ ,则:
$$
\begin{aligned}
f_{t,i,j}&=f_{t-1,i,j}+f_{t-1,i-1,j-p_t}\\
{i-1\choose k-1}&={i-2\choose k-2}+{i-2\choose k-1}\\
F_{t-1,j,k}&=\sum_{i=k}^{n}(-1)^{i-k}{i-1\choose k-1}f_{t-1,i,j}\\
F_{t-1,j-p_t,k-1}&=\sum_{i=k-1}^{n}(-1)^{i-k+1}{i-1\choose k-2}f_{t-1,i,j-p_t}\\
&=\sum_{i=k}^{n}(-1)^{i-k}{i-2\choose k-2}f_{t-1,i-1,j-p_t}\\
F_{t-1,j-p_t,k}&=\sum_{i=k}^{n}(-1)^{i-k}{i-1\choose k-1}f_{t-1,i,j-p_t}\\
&=\sum_{i=k+1}^{n}(-1)^{i-1-k}{i-2\choose k-1}f_{t-1,i-1,j-p_t}\\
&=-\sum_{i=k}^{n}(-1)^{i-k}{i-2\choose k-1}f_{t-1,i-1,j-p_t}\\
F_{t,j,k}&=F_{t-1,j,k}+F_{t-1,j-p_t,k-1}-F_{t-1,j-p_t,k}
\end{aligned}
$$


\subsection{二项式反演}

$$
f_n = \sum_{i=0}^n (-1)^i {n \choose i} g_i \Leftrightarrow g_n = \sum_{i=0}^n (-1)^i {n \choose i} f_i
$$
$$
f_n = \sum_{i=0}^n {n \choose i} g_i \Leftrightarrow g_n = \sum_{i=0}^n (-1)^{n-i} {n \choose i} f_i
$$

证明考虑直接代入。

\subsection{Min-Max容斥}

对于一个集合 $S$ ,定义 $\max(S)$ 为 $S$ 中最大值,$\min(S)$ 为 $S$ 中最小值,设:
$$
\max(S)=\sum_{T\subseteq S}f(|T|)\min(T)
$$

\vspace{-0.3cm}

设 $a_i$ 为 $S$ 中第 $i$ 大的数,则:

\vspace{-0.3cm}

$$
\max(S)=\sum_{i=0}^{|S|-1}a_{i+1}\sum_{j=0}^{i}f(j+1){i\choose j}=a_1
$$
$$
\sum_{j=0}^{i}f(j+1){i\choose j}=[i=0]
$$
$$
f(i+1)=\sum_{j=0}^{i}(-1)^{i-j}{i\choose j}[j=0]=(-1)^{i},f(i)=(-1)^{i-1}
$$
$$
\max(S)=\sum_{T\subseteq S}(-1)^{|T|-1}\min(T)
$$
同理,$\max_{k}(S)$ 为 $S$ 中第 $k$ 大值,则:
$$
f(i+1)=\sum_{j=0}^{i}(-1)^{i-j}{i\choose j}[j=k-1]=(-1)^{i-k+1}{i\choose k-1},f(i)=(-1)^{i-k}{i-1\choose k-1}
$$
$$
\max_k(S)=\sum_{T\subseteq S}(-1)^{|T|-k}{|T|-1\choose k-1}\min(T)
$$

$\min,\max$ 互换也成立(显然对称),期望意义下也成立。


\subsection{广义容斥}

求选出恰好 $K$ 个时的方案数(集合答案仅与大小有关时下面的推导才成立)。

设 $ans_S$ 表示选出集合为 $S$ 时的方案数,$ans_i$ 表示选出 $i$ 个的方案数。$a_i$ 表示至少选了 $i$ 个的方案数。
$$
ans_K=\sum_{S}f(|S|)a_S=\sum_{S}f(|S|)\sum_{S\subseteq T}ans_{T}=\sum_{T}ans_T\sum_{S\subseteq T}f(|S|)=\sum_{i=0}^{n}ans_i\sum_{j=0}^{i}{i\choose j}f(j)
$$
$$
\sum_{j=0}^{i}{i\choose j}f(j)=[i=K]
$$

\vspace{-0.4cm}

二项式反演:

\vspace{-0.4cm}

$$
f(n)=\sum_{i=0}^{n}(-1)^{n-i}{n\choose i}[i=K]=(-1)^{n-K}{n\choose K}
$$
$$
ans_K=\sum_{i=0}^{n}f(i)a_i=\sum_{i=K}^{n}(-1)^{i-K}{i\choose K}a_i
$$

\subsection{牛顿二项式定理}

$$
(x+y)^{\alpha}=\sum_{i=0}^{+\infty}{\alpha\choose i}x^iy^{\alpha-i}\ ,\ \ 
{\alpha\choose i}={\prod_{j=0}^{i-1}(\alpha-j)\over i!}
$$

其中 $\alpha$ 是实数。

\newpage

\subsection{序列容斥}

从一个位置走到下一个位置有若干种走法(合法或非法)。定义 $f_i$ 表示走了 $i$ 格均合法的方案数,$g_i$ 表示走了 $i$ 格均非法的方案数。特殊的,$g_1$ 也认为非法,$f_0$ 为初值。

先考虑前 $n-1$ 行都合法,则权值为 $f_{n-1}g_1$ ,这样有个问题,就是 $n-1$ 行到 $n$ 行时不一定合法,因此我们强制 $n-1$ 到 $n$ 非法,即减去 $f_{n-2}g_2$ ,但是类似的会发现 $f_{n-2}g_2$ 里包含了 $n-2$ 到 $n-1$ 非法的情况是我们多减的,所以加上 $f_{n-3}g_3$ 。以此类推会发现这是个容斥:
$$
f_{n}=\sum_{i=1}^{n}(-1)^{i-1}f_{n-i}g_i=\sum_{i=1}^{n}f_{n-i}(-1)^{i-1}g_{i}
$$
$$
h_{i}=(-1)^{i-1}g_i,f_n=\sum_{i=1}^{n}f_{n-i}h_i
$$
即 $f=f*h+f_{0},f={f_0\over 1-h}$ ,如果能求出 $g$ 即可求出 $f$ 。



\subsection{q-binomial}

$$
[n]_q = \sum\limits_{i=0}^{n-1} q^i = \lim_{x \rightarrow q} \frac{1-x^n}{1-x}, [ n ] !_q = \prod_{i=1}^n [i]_q, {n \brack m}_q = \frac {[n]!_q} {[m]!_q [n - m]!_q}
$$
$$
{n \brack m}_q = \lim_{x \rightarrow q} \frac{\prod_{1 \le i \le n}\frac{1-x^i}{1-x}}{\prod_{1 \le i \le m}\frac{1-x^m}{1-x}\prod_{1 \le i \le n-m}\frac{1-x^m}{1-x}} \\= \lim_{x \rightarrow q}\frac{\prod_{n-m+1 \le i \le n} (1-x^i)}{\prod_{1 \le i \le m} (1-x^i)}
$$

\vspace{0.4cm}

\textbf{结论1}:$\displaystyle {n \brack m}_q = {n-1 \brack m-1}_q + q^m {n-1 \brack m}_q=q^{n-m} {n-1 \brack m-1}_q + {n-1 \brack m}_q$

\vspace{0.2cm}

(1) 组合意义:${n \brack m}_q$ 表示 $(0,0)$ 向上向右走到 $(n-m,m)$ 的所有方案中 $q^{count}$ 的和,$count$ 表示路径右下角/左上角点的个数。

(2)说明:考虑坐标变换 $(n,m)\to(n-m,m)$ ,这样式子中就变成了向上向右走,然后考虑 $q^m$ 为每次向右走时下面点的个数,同理 $q^{n-m}$ 为每次向上走时左边点的个数。


\vspace{0.4cm}

\textbf{结论2}:$\displaystyle \prod_{i=0}^{n-1} (1+q^iy) = \sum\limits_{i=0}^n q^{\binom{i}{2}} {n \brack i}_q y^i$

\vspace{0.2cm}

(1) 组合意义:${n \brack i}_q$ 表示在长度为 $n$ 的序列里选 $i$ 个位置的所有方案中 $\prod_{j=1}^{i}q^{pre_j}$ 的和,$pre_j$ 表示第 $j$ 个选的位置前面有多少个没有被选。

(2) 说明:式子前面的生成函数表示第 $i$ 个位置选的贡献为 $q^i$ ,$y^i$ 的系数表示选了 $i$ 个位置的贡献乘积和。然后考虑 $q^{i\choose 2}$ ,表示每个选了的位置往前和其他选了的位置的贡献,因此考虑从 $y^i$ 的系数中去掉 $q^{i\choose 2}$ 就是 ${n \brack i}_q$ 的组合意义。

\vspace{0.4cm}

\textbf{结论3}:$\displaystyle {n + m \brack k}_q = \sum\limits_{i=0}^k q^{(n-i)(k-i)} {n \brack i}_q {m \brack k-i}_q$ ,证明考虑结论2组合意义,两边合并。

\textbf{结论4}:$\displaystyle {n + m + 1 \brack n + 1}_q = \sum\limits_{i=0}^m q^i {n + i \brack n}_q$ ,证明 $\displaystyle {n \brack m}_q=q^{n-m} {n-1 \brack m-1}_q + {n-1 \brack m}_q$ 展开 $m$ 次。

\textbf{结论5}:$\displaystyle \frac{1}{\prod_{i=0}^n (1-q^ix)} = \sum\limits_{i \ge 0} x^i {i+n \brack n}_q$

\newpage

\subsection{第一类斯特林数}

第一类斯特林数:$n$ 个数分成 $m$ 个圆排列,且没有空排列的方案数。
$$
{n\brack m}=(n-1){n-1\brack m}+{n-1\brack m-1}
$$
$$
x^{\overline{n}}=\prod_{i=0}^{n-1}(x+i)=\sum_{i=0}^{n}{n\brack i}x^i
$$
第二个式子:观察递推式可知另一个意义,对于 $[1,n]$ 的每个数都有 $1$ 种方案选和 $i-1$ 种方案不选,求选出 $m$ 个数的方案数,定义上升幂 $x^{\overline{n}}=\prod_{i=0}^{n-1}(x+i)$ ,则 $x^{\overline{n}}$ 第 $m$ 项系数即为 ${n\brack m}$ 。

\subsubsection{求一行}

求 ${n\brack 0},{n\brack 1},\cdots,{n\brack n}$ 。
$$
\begin{aligned}
x^{\overline{n}}&=\prod_{i=0}^{n-1}(x+i)=\sum_{i=0}^{n}{n\brack i}x^i\\
x^{\overline{2n}}&=x^{\overline{n}}(x+n)^{\overline n}\\
\end{aligned}
$$
$$
\begin{aligned}
(x+n)^{\overline n}&=\sum_{i=0}^{n}a_i(x+n)^i\\
&=\sum_{i=0}^{n}a_i\sum_{j=0}^{i}{i\choose j}x^jn^{i-j}\\
&=\sum_{i=0}^{n}x^i{1\over i!}\sum_{j=i}^{n}a_jj!{n^{j-i}\over(j-i)!}\\
\end{aligned}
$$
利用倍增求出 $x^{\overline{n}}$ 即可得出。

\lstinputlisting{./code/comb/stirling1row.cpp}

\newpage

\subsubsection{求一列}

求 ${0\brack m},{1\brack m},\cdots,{n\brack m}$ 。

考虑指数型生成函数,令 $\displaystyle F(x)=\sum_{i=1}^{+\infty}(i-1)!{x^i\over i!}$

则 $F(x)$ 的 $\displaystyle [{x^i\over i!}]$ 表示 $i$ 个数分成 $1$ 个圆排列的方案数。

那么 $F(x)^m$ 的 $\displaystyle [{x^i\over i!}]$ 表示 $i$ 个数分成 $m$ 个有顺序的圆排列的方案数。

\vspace{0.2cm}

所以 $\displaystyle {i\brack m}=[{x^i\over i!}]{F(x)^m\over m!}$ ,可以通过多项式快速幂求出(分治,或ln+exp)。

\vspace{0.3cm}

\lstinputlisting{./code/comb/stirling1column.cpp}

\subsection{第二类斯特林数}

第二类斯特林数:$n$ 个不同元素分成 $m$ 个相同集合,且没有空集的方案数。
$$
{n\brace m}=m{n-1\brace m}+{n-1\brace m-1}
$$
$$
{n\brace m}={1\over m!}\sum_{i=0}^{m}(-1)^i{m\choose i}(m-i)^n
$$
$$
n^k=\sum_{i=0}^{min\{n,k\}}{n\choose i}{k\brace i}i!
$$
第二个式子是个容斥,$i$ 枚举的是空盒子的个数,那么 $n$ 个球可以在其他盒子乱放,但由于第二类斯特林数表示的是相同盒子,所以最后要除以 $m!$ 。

第三个式子左边表示 $k$ 个不同的球放进 $n$ 个不同的盒子的方案数,右边枚举了非空盒子的个数 $i$ ,那么把 $k$ 个球放入这 $i$ 个盒子,由于第二类斯特林数表示的是相同盒子,所以还要乘上 $i!$ 。

\newpage

\subsubsection{求一行}

求 ${n\brace 0},{n\brace 1},\cdots,{n\brace n}$ 。
$$
{n\brace m}={1\over m!}\sum_{i=0}^{m}(-1)^i{m\choose i}(m-i)^n=\sum_{i=0}^{m}{(-1)^i\over i!}{(m-i)^n\over (m-i)!}
$$
显然是一个卷积。

\lstinputlisting{./code/comb/stirling2row.cpp}

\subsubsection{求一列}

求 ${0\brace m},{1\brace m},\cdots,{n\brace m}$ 。

考虑指数型生成函数,令 $\displaystyle F(x)=\sum_{i=1}^{+\infty}{x^i\over i!}=e^x-1$

则 $F(x)$ 的 $\displaystyle [{x^i\over i!}]$ 表示 $i$ 个不同元素放进 $1$ 个盒子的方案数。

那么 $F(x)^m$ 的 $\displaystyle [{x^i\over i!}]$ 表示 $i$ 个不同元素放进 $m$ 个不同盒子的方案数。

\vspace{0.2cm}

所以 $\displaystyle {i\brace m}=[{x^i\over i!}]{F(x)^m\over m!}$ ,可以通过多项式快速幂求出(分治,或ln+exp)。

\vspace{0.4cm}

\lstinputlisting{./code/comb/stirling2column.cpp}

\newpage

\subsection{斯特林科技}

\subsubsection{上升幂下降幂}

$$
\begin{aligned}
x^{\overline{n}} &=\prod_{i=0}^{n-1}(x+i)={x+n-1\choose n}n!\\
x^{\underline{n}} &=\prod_{i=0}^{n-1}(x-i)={x\choose n}n!\\
x^{\overline{n}} &=(-1)^{n}(-x)^{\underline{n}},x^{\underline{n}}=(-1)^{n}(-x)^{\overline{n}}\\
x^{\overline{n}} &=\sum_{i=0}^{n}{n\brack i}x^i,x^{\underline{n}}=\sum_{i=0}^{n}(-1)^{n-i}{n\brack i}x^i\\
x^n &=\sum_{i=0}^{n}{x\choose i}i!{n\brace i}=\sum_{i=0}^{n}{n\brace i}x^{\underline{i}}=\sum_{i=0}^{n}(-1)^{n-i}{n\brace i}x^{\overline{i}}
\end{aligned}
$$

\subsubsection{斯特林反演}

$$
\begin{aligned}
f_n &=\sum_{i=0}^{n}{n\brace i}g_i\Leftrightarrow g_n=\sum_{i=0}^{n}(-1)^{n-i}{n\brack i}f_i\\
f_n &=\sum_{i=0}^{n}{n\brack i}g_i\Leftrightarrow g_n=\sum_{i=0}^{n}(-1)^{n-i}{n\brace i}f_i
\end{aligned}
$$

\subsection{贝尔数}

$B_n$ 表示 $n$ 个有标号物品分成若干个非空无标号集合的方案数。

$$
\begin{aligned}
B_n &=\sum_{i=0}^{n}{n\brace i}=\sum_{i=0}^{n-1}{n-1\choose i}B_i\\
{B_n\over(n-1)!} &=\sum_{i=0}^{n-1}{1\over(n-1-i)!}{B_i\over i!}\\
\hat B(x)e^x &=\sum_{i=0}^{\infty}B_n{x^{n-1}\over(n-1)!}=\hat B'(x),\hat B(x)=\exp(e^x-1)
\end{aligned}
$$
