\section{数据结构}


\subsection{线段树}

\subsubsection{求上升序列}

初始 $val=-INF$ ,遍历 $\{a_n\}$ ,如果 $a_i\ge val$ ,更新 $val$ 为 $a_i$ ,问更新次数。

定义 $Find(p,k)$ 表示在 $p$ 节点前 $val=k$ 得到的 $val$ 变大的次数。

如果 $k>MAX_{ls_p}$ ,可以直接查询 $Find(rs_p,k)$ 。否则查询 $Find(ls_p,k)+Find(rs_p,MAX_{ls_p})$ 。

$Find(rs_p,MAX_{ls_p})$ 是一个固定值,记录下来之后每次就只会查询一边了。

\lstinputlisting{./code/ds/incseq.cpp}

\vspace{-0.5cm}

\subsubsection{线段树上二分}

\vspace{-0.2cm}

\lstinputlisting{./code/ds/segtrbinary.cpp}

\subsubsection{分裂合并}

\lstinputlisting{./code/ds/segtrmergesplit.cpp}

\vspace{-0.3cm}

\subsubsection{维护平方和以及历史平方和}

记 $len$ 表示区间长度,$sum$ 表示区间和,$val$ 表示区间平方和,$pre$ 表示到目前为止的前缀区间平方和,然后我们考虑区间 $+k$ 操作:
$x^2\to(x+k)^2=x^2+2kx+k^2$

\vspace{-0.4cm}

$$
\begin{aligned}
len&=len'\\
sum&=sum'+k\cdot len\\
val&=val'+2k\cdot sum'+k^2\cdot len\\
pre&=pre'+val'+2k\cdot sum'+k^2\cdot len\\
\end{aligned}
$$

\vspace{-0.2cm}

如果仅记录数值 $tag$ 我们不难发现没有办法进行标记合并,观察上面的式子发现这是一次线性变换,因此我们可以用矩阵来表示:
\vspace{-0.8cm}
$$
\begin{bmatrix}len'&sum'&val'&pre'\end{bmatrix}\begin{bmatrix}1&k&k^2&k^2\\0&1&2k&2k\\0&0&1&1\\0&0&0&1\end{bmatrix}=\begin{bmatrix}len&sum&val&pre\end{bmatrix}
$$

\vspace{-0.2cm}

所以我们可以将这个矩阵记为 $tag$ ,这样就能够标记合并了。

\newpage

这样常数很大,由于这个转移矩阵是上三角的,因此矩阵乘法时我们只需要枚举上三角,常数小很多。

实现时需要注意,为了对齐时间(求 $pre$ ),不仅需要对 $[L,R]$ 进行 $+k$ 操作,还需要对 $[1,L-1],[R+1,n]$ 进行 $+0$ 操作。

可把 $len$ 定义改为 $si$ 表示区间中存在点的个数,从而解决某些问题。

\lstinputlisting{./code/ds/segtrsqrsum.cpp}

\subsubsection{李超线段树}

对于每个线段树节点,记录“覆盖最广线段” $(k,b)$ (顾名思义就是作为最高线段的区间最长的线段,不难发现询问 $x=k$ 的答案一定在包含 $k$ 的节点的“覆盖最广线段”中)。然后对于一个节点 $old$ ,当有新的线段 $new$ 出现的时候,我们需要修改标记:

交点在左边的时候,说明 $new$ 是“覆盖最广线段”,先把 $old$ 当做新的线段去修改左边,然后将 $old$ 修改为 $new$ 。

交点在右边的时候,说明 $old$ 是“覆盖最广线段”,把 $new$ 当做新的线段去修改右边。

\lstinputlisting{./code/ds/lichaosegtr.cpp}

\newpage

\subsubsection{主席树求mex以及所有子区间mex的mex}

开 $n$ 个权值主席树,第 $i$ 棵树记录 $1\sim i$ 中 $x$ 最后一次出现的位置 $pos_x$ 。询问 $[L,R]$ 中的 $mex$ 时,在第 $R$ 棵树中查询,如果左儿子最小的 $pos<L$ ,就说明左儿子中有数没有出现,因此往左儿子查找,否则往右儿子查找。

对于 $x$ ,我们以 $a_i=x$ 的点为分割点,把数组分成好多块,如果任意一个块的 $mex$ 等于 $x$ ,就说明 $x$ 合法。

最后求所有合法 $x$ 的 $mex$ 就行了。

(示例代码中mex定义为最小正整数,定义为自然数时需要更改权值线段树边界)

\lstinputlisting{./code/ds/chairsegtrmex.cpp}

\newpage


\vspace{-0.1cm}

\subsection{LCT}

\subsubsection{正常版}

\lstinputlisting{./code/ds/LCT.cpp}

\vspace{-0.5cm}

\subsubsection{缩边双版}

\lstinputlisting{./code/ds/exLCT.cpp}

\subsubsection{维护子树信息}

LCT实链上的信息可以方便维护,所以我们关注虚边:虚边只在Access和Link的时候改变。那么我们只需要记录两个信息 $si[x][0],si[x][1]$ 表示虚边节点数以及虚边节点数加上Splay中子树节点数,则 $si[x][1]=si[L[x]][1]+si[R[x]][1]+si[x][0]+1$ 。然后Access和Link的时候顺便维护一下 $si[x][0]$ 就行了。注意Link的时候一定要让被更新节点成为根,否则无法更新信息(将引起大量信息改动)。

\vspace{0.2cm}

树合并,维护重心:维护出子树信息之后,根据重心的性质,新的树的重心在原来两棵树重心的路径上,将这条路径取出之后在Splay上二分,每次看左右两边子树大小是否 $\le {si\over 2}$ ,如果满足则是重心。每次往节点多的那边走。

\lstinputlisting{./code/ds/subLCT.cpp}

\vspace{-0.3cm}

\subsubsection{维护置换群}

给出置换 $p_i$ ,每次置换执行 $a_i=a_{p_i}$ 。环断链,LCT维护每个置换群的头尾。支持交换 $p_x,p_y$ 。

\lstinputlisting{./code/ds/circleLCT.cpp}

\subsection{Splay维护区间}

\lstinputlisting{./code/ds/Splay.cpp}

\vspace{-0.4cm}

\subsection{KDT}

\vspace{-0.3cm}

各种估价函数看Claris板子。

\lstinputlisting{./code/ds/KDT.cpp}

\subsection{猫树}

定义 $\displaystyle A_{i,j}=\prod_{k=\lfloor{i\over 2^j}\rfloor2^j}^{i}a_k,B_{i,j}=\prod_{k=i}^{\lceil{i\over 2^j}\rceil2^j-1}a_k$ ,可以 $O(n\log_2n)$ 处理。

然后询问区间 $[L,R]$ ,如果 $L=R$ 那么就是 $a_L$ ,否则我们需要找到一个 $k$ 使得 $[L,R]$ 内只存在一个 $2^k$ 的倍数,这样的话就可以直接 $B_{L,k}\cdot A_{R,k}$ 得出结果。

找规律可知 $highbit(L\ xor\ R)$ 就是一个合法的 $k$ 。

\lstinputlisting{./code/ds/cattree.cpp}

\subsection{ODT}

用set维护相同信息的线段。若数据含有Assign并且随机,则块数是 $O(\log\log n)$ 的。

区间操作时暴力遍历 $[L,R]$ 中所有线段。

\lstinputlisting{./code/ds/ODT.cpp}

\newpage

\subsection{随机堆}

\lstinputlisting{./code/ds/randomheap.cpp}


