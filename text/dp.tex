\section{动态规划}

\subsection{背包回退}

\lstinputlisting{./code/dp/bagback.cpp}

\subsection{树形DP}

\subsubsection{树形背包优化}

\lstinputlisting{./code/dp/magictreebag.cpp}

\subsubsection{路径有方向}

求一条最优路径,但如果一条边行走方向不同,产生的权值不同。

定义 $f_i$ 表示向上走到 $i$ 的状态,$g_i$ 表示 $i$ 向下走的状态。

DFS时正着枚举儿子做一遍,倒着枚举儿子做一遍,即可考虑到两个儿子 $u\to v,v\to u$ 两种方向。

\subsubsection{DFS序上DP}

树形背包,每个节点有 $s_x$ 个体积 $w_x$ 价值 $p_x$ 的物品。

但需要满足儿子选了至少一个的话,则父亲也必须至少选一个。

定义 $f_{i,j}$ 表示DFS序 $[1,i]$ 体积为 $j$ 的最优解,这样转移就只有两种:

1. $i$ 点选,那么转移到 $f[i+1][k]$ 。

2. $i$ 点不选,那么子树也没有选,转移到 $f[out[i]+1][k]$ 。

\lstinputlisting{./code/dp/dfsorderdp.cpp}

\subsection{状压DP}

\subsubsection{斯坦纳树}

\textbf{一般问题}

在 $n\times m$ 的网格上有 $k$ 个景点,选取每个网格都有代价,求最小代价使得景点之间全部连通。

定义 $f_{i,j,s}$ 表示当 $i,j$ 与 $s$ 这些景点连通时的最优解,那么有两种方法转移:

1. 通过子集转移:$f_{i,j,t}+f_{i,j,s-t}-cost_{i,j}\to f_{i,j,s}$ ,减去 $cost_{i,j}$ 是因为算重复了。

2. 通过相邻点转移:$f_{x,y,s}+cost_{i,j}\to f_{i,j,s}$ 。

第二种转移是最短路形式,用SPFA进行,每次先用第一种转移处理出 $f_{i,j,s}$ ,然后把 $f_{i,j,s}$ 加进队列,对每个 $s$ 都做一遍SPFA。

\lstinputlisting{./code/dp/steinertree.cpp}

\textbf{随机映射}

$n\times m$ 网格,每个格点颜色为 $c_{i,j}$ ,权值 $cost_{i,j}$ 。选取一个最小权值连通块使得至少选了 $K$ 种颜色。

若 $K$ 小,颜色集大,可以将颜色集随机映射到 $[1,K]$ 内,然后做斯坦纳树,这样单次正确概率为 $\frac{K!}{K^K}$ 。

\subsubsection{DP套DP}

给定一个状态 $S$ ,可用DP $f$ ,求其最优解为 $f(S)$ 。问多少个 $S$ 满足 $f(S)=F$ 。

将 $f$ 的DP状态记录下来,可以得到一个DAG,外层DP储存在DAG的位置。

第 $a$ 步,当状态 $A$ 添加元素时,即从 $f(A)\to f(B)$ 。

将 $f(A)$ 的DP状态压缩为 $s_A$ ,令外层DP $g_{a,s_A}$ ,则 $g_{a,s_A}\to g_{a+1,s_B}$ 进行转移。

\newpage

\subsubsection{插头DP}

记录轮廓线上的插头状态,处理当前格子时将左上角轮廓更新为右下角轮廓。

\vspace{0.3cm}

\textbf{哈密顿回路}

任意条哈密顿回路铺满网格:记录 $0,1$ 表示是否有插头。

一条哈密顿回路铺满网格:记录 $0,1,2$ 表示没有插头,左括号插头,右括号插头。

\subsection{斜率优化}

若DP在 $j$ 比 $k$ 优秀时满足以下式子( $j<i,k<i,X_k<X_j$ ):

$$
f_j+val(j)-[f_k+val(k)]\le A_i(X_j-X_k) \Rightarrow
\frac{Y_j-Y_k}{X_j-X_k}\le A_i
$$

记 $K(a,b)=\frac{Y_b-Y_a}{X_b-X_a}(X_a<X_b)$ ,则 $K(k,j)\le A_i$ 说明 $j$ 比 $k$ 优秀,$K(k,j)>A_i$ 说明 $k$ 比 $j$ 优秀。

推论:若 $K(k,j)\ge K(j,t)$ ,则 $j$ 对任何 $i$ 都必然不优秀。

证明:若 $K(j,t)\le A_i$ ,则 $t$ 比 $j$ 优秀。若 $K(j,t)>A_i$ ,则 $K(k,j)>A_i$ ,$k$ 比 $j$ 优秀。

因此有用的点必须满足 $K(que_{i-1},que_i)<K(que_i,que_{i+1})$ ,即 $que$ 构成一个下凸壳。

\subsubsection{队列维护}

若 $X_i,A_i$ 均递增,可以用队列维护凸壳。

更新点 $i$ 时,判断队首,如果 $K(que_{head},que_{head+1})\le A_i$ 则 $que_{head}$ 失效(因为 $A_i$ 只会增大)。

更新完成后 $que_{head}$ 即最优解。

加入点 $i$ 时,判断队尾,如果 $K(que_{tail-1},que_{tail})\ge K(que_{tail},i)$ 则队尾失效。

\subsubsection{cdq分治}

若 $X_i,A_i$ 有非递增,可以用cdq分治。

求出 $[L,mid]$ 的答案后,构造出左边的凸壳,然后在凸壳上二分斜率,更新 $[mid+1,R]$ 的答案。

当然也可以平衡树动态维护凸壳。

\lstinputlisting{./code/dp/cdqslopeoptimize.cpp}

\subsection{决策单调性}

若DP对于任意的 $i<j<k<t$ 且对于 $k$ 来说 $j$ 比 $i$ 优秀,都满足对于 $t$ 来说 $j$ 比 $i$ 优秀,则该DP满足决策单调性。

令 $pos_i$ 表示转移到 $i$ 的最优位置,则 $pos$ 是单调不降的(因为取到 $pos_i$ 说明前面都没有 $pos_i$ 优秀,那么对于 $i$ 后面的肯定也是不优秀的)。

维护队列,队列元素 $p_i,l_i,r_i$ 表示 $[l_i,r_i]$ 的 $pos$ 都为 $p_i$ ,每次考虑加入 $i$ 这个决策点,由于 $pos$ 单调,因此可以二分找到从哪个位置开始 $pos$ 变为 $i$ 。

\lstinputlisting{./code/dp/decisionmonotonicity.cpp}