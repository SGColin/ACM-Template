\section{字符串}

\subsection{Trie}

\subsection*{倍增技巧}

如果从上往下一直跳某个字符,每个节点需要对于每个字符都开倍增数组,是不可接受的。如果我们记录 $who_{x,ch}$ 表示节点 $x$ 往下一直跳 $ch$ 字符最深能到哪里,就可以通过从下往上跳规避掉这个问题。

\subsection*{一种复杂度正确的暴力做法} 

Trie可以对于每个节点记录 $O(size)$ 大小的节点信息。

每次插入一个数,只会让串长个节点的 $si$ 增加,也就是说,元素个数总和为 $O(\sum len)$ 。


\subsection{Manacher}

\lstinputlisting{./code/string/manacher.cpp}



\subsection{AC自动机}



\lstinputlisting{./code/string/acam.cpp}



\subsection{Z 函数}



z[i] 是从 i 开始的后缀与整个串最长公共前缀的长度。

\begin{lstlisting}
int l = 0, r = 0;
for (int i = 2; i <= n; ++i) {
    if (i <= r) z[i] = min(z[i - l + 1], r - i + 1);
    while (i + z[i] <= n && t[z[i] + 1] == t[i + z[i]]) ++z[i];
    if (i + z[i] - 1 > r) {l = i; r = i + z[i] - 1;}  
}
\end{lstlisting}

\newpage

\subsection{Lyndon分解}

$S$ 是Lyndon串当且仅当 $S$ 本身是所有后缀中的最小串。$S$ 是Lyndon串可以推出 $S$ 是所有循环表示中字典序最小的。各种性质与推论:

1. $A,B$ 都是Lyndon串且 $A<B$ ,则 $A+B$ 也是Lyndon串,因为 $B>A+B$ 。

2. 如果 $S=A+B$ 是Lyndon串,则 $B>S>A$ 。

3. $S$ 可以唯一划分成 $A_1A_2\cdots A_m$ , $A_i$ 均为Lyndon串,且 $A_i\ge A_{i+1}$ ,划分方法即Lyndon分解。

4. 若Lyndon串 $S=A+c+B$ ,则 $A+d$ 是Lyndon串(其中 $c,d$ 为字符,且 $d>c$ )。

5. Lyndon串没有Border,所以Lyndon串也没有周期。

\textbf{Duval算法}

维护三个位置 $i,j,k$ ,$S[1,i-1]$ 已经分解完毕。

现在 $S[i,k-1]=A^{t}+B$ ,其中 $A$ 是Lyndon串,$A^t$ 是多个 $A$ 接一起。

$B$ 是 $A$ 的可空前缀,$j=k-|A|$ 表示 $S_k$ 在 $A$ 中的对应位置,接下来考虑加上 $k$ 这个位置:

1. $S_k=S_j$ ,则 $A$ 仍是周期,$j,k$ 均往后移动。

2. $S_k>S_j$ ,由性质3 $B+k$ 是Lyndon串且 $A<B+k$ ,前面全部合并,新的 $A$ 更新为 $S[i,k]$ 。

3. $S_k<S_j$ ,则 $A>B+k$ ,所有的 $A$ 都已经确认分解完毕,将 $i$ 移动到 $B$ 的开头。

\begin{lstlisting}
for (int i=1;i<=n;){
	int j=i,k=j+1;
	for (;k<=n && s[k]>=s[j];k++) s[k]==s[j]?j++:j=i;
	for (;i<=j;i+=k-j) L[++m]=i,R[m]=i+(k-j)-1;
}
\end{lstlisting}

\textbf{最小表示法}

把串再接一遍,最后一个 $L_i\le n$ 的Lyndon串就是最小的循环串。

这样求出的是最大位置,如果要最小位置,对于每次的循环串只取第一个 $i$ 即可。

\begin{lstlisting}
for (int i=1;i<=n;){
	int j=i,k=j+1;
	for (;k<=n && s[k]>=s[j];k++) s[k]==s[j]?j++:j=i;
	for (;i<=j;i+=k-j) if (i<=(n>>1)) pos=i;
}
\end{lstlisting}

\textbf{前缀的最小后缀}

对于一个前缀,Lyndon分解之后,最小后缀就是最后一个Lyndon串。

\begin{lstlisting}
for (int i=1;i<=n;){
	int j=i,k=j+1; ans[i]=i;
	for (;k<=n && s[k]>=s[j];k++) 
		if (s[k]==s[j]) ans[k]=ans[j]+k-j,j++; else ans[k]=i,j=i;
	for (;i<=j;i+=k-j);
}
\end{lstlisting}

\textbf{前缀的最大后缀}

字符集翻转后对于前缀进行Lyndon分解,如果最大后缀不是最后一个Lyndon串,则一定是将这个Lyndon串作为前缀的串。

\begin{lstlisting}
for (int i=1;i<=n;){
	int j=i,k=j+1; if (!ans[i]) ans[i]=i;
	for (;k<=n && s[k]<=s[j];k++){if (!ans[k]) ans[k]=i; s[k]==s[j]?j++:j=i;}
	for (;i<=j;i+=k-j);
}
\end{lstlisting}

\newpage

\subsection{后缀数组}

\subsubsection{倍增}

\lstinputlisting{./code/string/SA.cpp}

\subsubsection{SA-IS}

\lstinputlisting{./code/string/SAIS.cpp}

\newpage

\subsection{(广义)后缀自动机}

\subsubsection{静态}

\lstinputlisting{./code/string/SAM.cpp}


\subsubsection{动态(LCT维护parent树)}

\lstinputlisting{./code/string/SAMLCT.cpp}

\subsection{回文自动机}

\subsubsection{基础}

\lstinputlisting{./code/string/PAM.cpp}

\vspace{-0.2cm}

\subsubsection{前端插入 与 记忆化}

\lstinputlisting{./code/string/exPAM.cpp}

\subsubsection{Palindrome Series}

在PAM上额外维护以下信息:

1. $dif_x=len_x-len_{fail_x}$ ,即 $x$ 与 $fail_x$ 最长Border的差,也就是所在等差数列的公差。

2. $anc_x$ 表示 $x$ 往上遇到的第一个 $dif_p\not=dif_{fail_p}$ 的 $p$ 的父亲,即所在等差数列的顶点的父亲。

3. $sum_x$ 表示 $[x,anc_x)$ 路径上的当前信息(包含 $x$ ,不包含 $anc_x$ )。

Palindrome Series 结论:

1. 一个串的长度在 $[2^k,2^{k+1})$ 范围内的Border长度构成等差数列(Border Series)。

2. 字符串 $T$ 是回文串 $S$ 的一个回文后缀,当且仅当 $T$ 是 $S$ 的Border。

3. 字符串 $S$ 存在回文Border $T$ ,且 $|T|\ge{|S|\over 2}$ ,则 $S$ 是回文串。

4. 如果 $dif_x>{len_x\over 2}$ ,则 $anc_x=fail_x$ 。

5. 如果 $dif_x=dif_{fail_x}$ ,则 $len_{fail_x}\ge{len_x\over 2}$ 。

6. 按照字符串前缀插入,则更新信息时,需要用到的 $sum_{fail}$ 一定已经被更新过了。

7. $len_{anc_x}\le{len_x\over 2}$ ,$x$ 沿着 $anc_x$ 只会跳 $\log_2n$ 次。

Palindrome Series 储存信息:
$$
\begin{aligned}
f_i&=\sum\{f_j|S[j+1,i]\text{ is Palindrome}\}\\
sum_x&=\sum\{f_{i-len_x},f_{i-len_{fail_x}},\cdots,f_{i-len_t}\},fail_t=anc_x\\
sum_{fail_x}&=\sum\{f_{j-len_{fail_x}},\cdots,f_{j-len_t}\},fail_t=anc_x\\
j&=i-dif_x,len_x=len_{fail_x}+dif_x\Rightarrow i-len_x=j-len_{fail_x}\\
sum_x&=sum_{fail_x}+f_{i-len_t},fail_t=anc_x
\end{aligned}
$$

适用于枚举回文后缀统计信息的题。

\lstinputlisting{./code/string/PAMPS.cpp}

