\section{树论}

\subsection{Prufer序列}

\subsubsection*{编码}

找到编号最小的度为 $1$ 的节点,删除该点及连接该点的边,将与该点相邻的点加入序列。

重复删除/加入,直到剩下两个节点,最后的序列就是Prufer编码(长度为 $n-2$ )。

\subsubsection*{解码}

记录一个集合 $S$ ,$S$ 刚开始是全集。

令 $x$ 是 $S$ 中在当前Prufer编码中没出现的最小编号,将 $x$ 与序列第一项连边,删除第一项以及 $x$ 。

重复连边/删除,直到序列为空,此时 $S$ 中还有两个节点,进行连边。

\subsubsection*{性质}

如果一个节点的度数为 $D$ ,那么该节点将会在Prufer编码中出现 $D-1$ 次。

\subsection{树链合并}

在线可以树剖+打覆盖标记线段树两个 $\log$ 。下面讲离线。

树上给 $n$ 条到根的链,每条链将路径上点标记,求有几个标记点。

将 $pos$ 按入栈序排序,把 $pos_j$ 到根路径上的点都 $+1$ ,把 $LCA(pos_{j-1},pos_j)$ 到根路径上的点都 $-1$ 。

多次树链合并,用差分+树状数组就可以单点查询。

\lstinputlisting{./code/graph/treechainmerge.cpp}

\subsection{直径合并}

两棵树合并的时候,新树直径的两个端点只会在两棵树直径的四个端点中产生。

\subsection{虚树}

\lstinputlisting{./code/graph/VirtualTree.cpp}

\subsection{树重心的性质}

1. 以重心为根,所有的子树的大小都不超过整个树大小的一半。

2. 一棵树最多只有两个重心。

3. 树中所有点到某个点的距离和中,到重心的距离和是最小的,如果有多个重心,他们的距离和一样。

4. 把两棵树通过某一点相连得到一颗新的树,新的树的重心必然在连接原来两棵树重心的路径上,

\hspace{0.3cm} 且这条路径上的点的最大子树在左右两个相邻的点上。

5. 一棵树添加或者删除一个节点,树的重心最多只移动一条边的位置。

\subsection{点分治}

模板:求树上距离为 $d$ 的点对数 $sum_d$ 。

\lstinputlisting{./code/graph/pointdandcFFT.cpp}

\subsection{点分树}

将点分治DFS的顺序建成一棵新树,新树高是 $\log_2n$ 的。假设 $x$ 节点在点分树上子树大小为 $size_x$ ,则对所有 $x$ 都可以记录 $O(size_x)$ 的信息(vector,线段树动态开点等),从而支持暴力信息修改。

\vspace{0.1cm}

点分树中通常会记录 $A_x$ 表示 $x$ 点分树子树中,与 $x$ 的原树信息。以及 $B_x$ 表示 $x$ 点分树子树中,与点分树父亲 $fa_x$ 的原树信息。以便进行容斥(和点分治同理)。


\vspace{0.1cm}

点分树中各种子树信息、距离信息基本和原树无关,如 $x$ 到 $fa_x$ 的距离不一定小于 $x$ 到 $fa_{fa_x}$ 的距离。

\vspace{0.2cm}

模板:求 $x$ 附近距离小于等于 $K$ 的点的权值和,带单点修改。


\vspace{0.1cm}

解法:维护 $A_x$ 表示 $x$ 点分树子树中,到 $x$ 距离的权值线段树。$B_x$ 表示 $x$ 点分树子树中,到 $fa_x$ 距离的权值线段树。首先查询 $A_x$ 线段树中距离 $[0,K]$ 的权值,然后暴力枚举 $x$ 的点分树祖先 $y$ ,查询 $A_{fa_y}$ 中 $[0,K-dis(x,fa_y)]$ 权值减去 $B_y$ 中 $[0,K-dis(x,fa_y)]$ 权值。修改也是暴力枚举 $x$ 的点分树祖先,修改 $A_y$ 和 $B_y$ 。

\lstinputlisting{./code/graph/pointdandctree.cpp}

\subsection{树链剖分}

\subsubsection{重链剖分}

求 $k$ 级祖先:跳到 $k$ 级祖先所在长链,然后通过深度和DFS序确定 $k$ 级祖先。

\subsubsection{长链剖分}

性质:$x$ 的 $k$ 级祖先所在长链长度一定 $\ge k$(否则显然有更长的链)。

\vspace{0.1cm}

求 $k$ 级祖先:倍增预处理出 $2^i$ 级祖先。先跳到 $2^{\lfloor\log_2k\rfloor}$ 级祖先所在长链的链顶 $y$ ,$y$ 所在长链长度大于等于 $2^{\lfloor\log_2k\rfloor}$ ,因此 $y$ 沿着长链往下或者往上 $len_y$ 个就能找到 $k$ 级祖先,对于每个链顶可以预处理这 $2len_y$ 个节点。

\vspace{0.2cm}

有关深度的DP:类似 $f_{i,j}$ 表示 $i$ 子树中与 $i$ 距离为 $j$ ,距离为 $d-j$ 等DP。

长链继承信息,短链暴力合并,复杂度 $O(n)$ 。

\vspace{0.1cm}

贪心:选 $k$ 个叶子到根的路径,求覆盖的边权和的最大值。

带权长链剖分之后,显然分成了若干个不相交的,到叶子的路径,选前 $k$ 大即可。

\lstinputlisting{./code/graph/lsd.cpp}

\subsection{链分治}

\subsubsection{重链分治}

将根节点所在重链视为第一层,第一层点连出去的轻链一定是另一个重链的顶点,作为第二层重链。

以此类推,由于轻儿子 $size$ 至少减半,因此这样分层只有 $\log_2n$ 层。

常用于处理子树DP的优化,例如动态最大独立集,独立集背包。

先求出轻链的信息,挂在重链节点上,然后对重链进行分治等方法快速合并信息。

\lstinputlisting{./code/graph/hlddandc.cpp}

\subsubsection{长链分治}

同重链分治分层方式,由于上层比下层至少长 $1$ ,因此只有 $2\sqrt n$ 层。

问题:每个节点给出多项式 $a_ix+b_i$ ,求所有叶子到根多项式乘积的和。

每个节点记录长链上多项式,以及拐到短链之后的多项式的和。

类似重链分治,先求出短链的多项式,加到长链上,然后对长链进行分治NTT,得到长链多项式,以及长链拐出去的多项式之和。

\newpage

\subsection{DSU on Tree}

按子树 size 轻重链剖分,然后使用某个数据结构统计每个点的子树信息:

1. 先让轻儿子统计子树信息,并撤销对数据结构的影响;

2. 如果有重儿子,统计重儿子子树信息,保留对数据结构的影响(不撤销);

3. 向数据结构中添加轻儿子子树信息和当前点信息,统计当前点信息。

\vspace{0.2cm}

复杂度分析:每个点只会在到根路径上遇到轻边时被添加 / 撤销,由轻重连剖分结论,每个点到根的路径上至多 $\log n$ 条轻边。所以总复杂度是  $\mathcal O(n \log n )$ ,由于明显跑不满所以常数会很小。

\vspace{0.2cm}

如果需要多次运行 DSU on Tree,可以把所有轻儿子的子树点集预处理出来,省掉递归的常数。

\lstinputlisting{./code/graph/dsu.cpp}

\newpage

\subsection{仙人掌}

\subsubsection{建图}

DFS树:在找到返祖边 $u\to x$ 时,建立一个方点,将 $x$ 连向方点,然后从 $u$ 暴力向上跳,将方点连向途中的点,并且记录环的总距离。求最短路:圆方边的边权是点到祖先点的最小距离(环上有两个方向),用环的总距离以及DFS树上的距离就可以算出。

询问 $x,y$ 时,在圆方树上求出 $x,y$ 的 $lca$ ,然后讨论:

1. $lca$ 是方点,先将 $x,y$ 跳到 $lca$ 的前一个节点 $fx,fy$ ,$x,y$ 走到 $fx,fy$ 之后再求环上的最小距离

2. $lca$ 是圆点,那么答案就是 $x,y$ 在圆方树上的距离

\lstinputlisting{./code/graph/Cactus.cpp}

\newpage

\subsubsection{最大独立集}

建出圆方树,定义 $f_{i,0/1}$ 表示 $i$ 点是否选择的最大独立集,并且对于方点,将 $f_{i,0/1}$ 定义为与方点 $i$ 的顶点相邻的节点是否选择的最大独立集。

那么圆点就正常进行最大独立集DP,而方点需要进行进一步DP。

由于DFS树建圆方树时是按照深度顺序建边的,因此对于方点,按照加边顺序获取出所有儿子即为环上的顺序。

定义 $F_{i,0/1,0/1}$ 表示前 $i$ 个节点中,第一个节点是否选择,且第 $i$ 个节点是否选择的最大独立集。

最后 $\max\{f_{1,0},f_{1,1}\}$ 就是答案。

\lstinputlisting{./code/graph/CactusDP.cpp}

\newpage

\section{图论}

\subsection{差分约束系统}

$a_y-a_x\le b$ 转化为 $x\to y$ 边长为 $b$ 的边,然后刷最短路,如果从 $S$ 开始,则 $dis_i-S$ 都是最大的(最短路顶到限制)。

同理 $a_y-a_x\ge b$ 转化为 $x\to y$ 边长为 $b$ 的边,刷最长路,如果从 $S$ 开始,则 $dis_i-S$ 都是最小的。

\vspace{-0.2cm}

\subsection{K短路}

每个点的代价函数为 $f(x)+h(x)$ ,其中 $f(x)$ 为 $S$ 到 $x$ 距离,$h(x)$ 为 $x$ 到 $T$ 距离(完美估价函数)。

使用A*,优先队列存 $f(x)+h(x)$ 最小的 $x$ ,每一次 $x$ 抵达 $T$ 就说明是比上一次劣的最优最短路。


\vspace{-0.2cm}

\subsection{二分图}

\subsubsection{生成树个数}

左 $n$ 个点,右 $m$ 个点的完全二分图,生成树个数为 $m^{n-1}\cdot n^{m-1}$ 。

\vspace{-0.2cm}

\subsubsection{霍尔定理}

二分图 $X,Y$ 存在完美匹配等价于:

对于 $X$ 的任意子集 $S$ ,定义 $M(S)$ 表示 $S$ 连向的 $Y$ 的点集,均满足 $|S|\le |M(S)|$ 。

\vspace{-0.2cm}

\subsubsection{稳定婚姻}

有 $n$ 对CP,和 $m$ 对前男女友关系。一对CP解散之后双方可能会和前男女友旧情复燃,导致很多CP都解散。问第 $i$ 对CP解散之后还能否使得所有人都找到新CP。

如果第 $i$ 对CP除了直接相连的边之外还能找到另外的增广路,说明就可以CP重组。那么把CP边建成 $(x,y)$ ,而前男女友关系建成 $(y,x)$ ,有多条增广路其实就是有SCC,Tarjan就好了。

不能双向边刷边双,这样可能会把前男女友与CP搞混。

\vspace{-0.2cm}

\subsubsection{König 定理}

二分图最大匹配大小 = 最小点覆盖大小。

构造最小点覆盖:假设此前求最大匹配是每次从左侧开始尝试增广,那么增广结束之后,图上没有增广路了。接下来,从左侧每一个未匹配点开始走交错路(匹配边,未匹配边间隔放置),经过的点都打上标记,复杂度 $\mathcal{O}(m)$。最小点覆盖就是\textbf{所有左侧未标记点和右侧标记点}。

任何图最大独立集和最小匹配为对偶问题(答案互补)。

\vspace{-0.2cm}

\subsubsection{Dilworth 定理}

对于偏序关系,最长反链(最大不可比较集)大小 = 最小链(可比较集)覆盖大小(链数)。

对偶定理成立:最小反链覆盖大小 = 最长链大小。

\vspace{0.2cm}

求最小链覆盖:先将偏序的 DAG 传递闭包,然后问题变成求解最小不可重路径覆盖。只需要将每个点拆成出点和入点,然后对于原图的边连 $u_{出}\to v_{入}$ ,答案即 $|V|-$ 最大匹配。

\vspace{0.2cm}

求最长反链:按上述构造二分图并求最大独立集,如果一个点的出入点都在其中,将其加入最长反链。

\subsubsection{Hopcroft-Karp}

Dinic 在二分图上的特殊化,求最大匹配复杂度 $\mathcal{O}(m\sqrt n)$ ,实际很难跑满,常数非常小。

使用时记得设置左右点数 $nl$ 和 $nr$ ,以及增广路长度上限 $inf$ 。

\newpage

\lstinputlisting{./code/graph/hopcroft_karp.cpp}

\newpage

\subsection{强连通分量}

\subsubsection{Kosaraju's Algorithm}

\noindent 
DFS1 : DFS \textbf{原图}记录后序 post order ,post number 最大的点一定在反图的某个汇中。

\noindent 
DFS2 : 按 DFS1 中 post order 的\textbf{逆序} DFS \textbf{反图},每次找到一个反图的汇 SCC ,从小到大编号。

\lstinputlisting{./code/graph/scc_kosaraju.cpp}

\subsubsection{Tarjan's SCC Algorithm}

\noindent 
$low[u]$ : 节点 $u$ 能到达的,且能走回来的最小的 $idx$ (强制在栈中)。

\noindent 
在\textbf{反图}上求 SCC 并缩点,则编号 $bl[u]$ 同原图拓扑序\footnote{证明:考虑原图中两个强连通分量 $S_1$ 和 $S_2$ ,若 $S_1$ 中某个点有一条指向 $S_2$ 中某个点的边,讨论反图中 $S_1$ 和 $S_2$ 的 DFS 顺序即可证明},即:若原图中 $u\to v$ ,则 $bl[u] \le bl[v]$ 。

\lstinputlisting{./code/graph/scc_tarjan.cpp}

\newpage

\subsubsection{缩点(去重边)}

任何有向图都是其强连通分量( SCC )的有向无环图( DAG )。

按上述代码编号,则编号 $bl[u]$ 从小到大即原图拓扑序:若原图中 $u\to v$ ,则 $bl[u] \le bl[v]$ 。

\lstinputlisting{./code/graph/scc_shrink.cpp}

\vspace{-0.4cm}

\subsubsection{构造强连通图方法}

首先,收缩 SCC. 假设有 $s$ 个源 (入度为 $0$ 的点) $v_0, \ldots, v_{s-1}$ 和 $t$ 个汇(出度为 0 的点) $w_0, \ldots, w_{t-1}$.

不失一般性,假设 $s \leq t$. 我们之后可以证明, $v_0, \ldots, v_{s-1}$ 和 $w_0, \ldots, w_{t-1}$ 可以重新排列,使得存在一个 $1 \leq p \leq s$ ,满足以下三个条件:

1. 对于 $i \in[0, p), v_i \rightsquigarrow w_i$.

2. 对于 $i \in[p, s)$, 存在 $j \in[0, p), v_i \rightsquigarrow w_j$.

3. 对于 $j \in[p, t)$, 存在 $i \in[0, p), v_i \rightsquigarrow w_j$.

其中, $u \rightsquigarrow v$ 代表存在一条从 $u$ 到 $v$ 的\textbf{路径}。

现在先假设满足上述性质的 $p$ 存在。连下列 $t$ 条边:

1. 对于 $i \in[0, p)$ ,连 $w_i \rightarrow v_{(i+1) \bmod p}$. 这样可以把 $v_0, \ldots, v_{p-1}, w_0, \ldots, w_{p-1}$ 变成一个环。

2. 对于 $i \in[p, s)$ ,连 $w_i \rightarrow v_i$. 因为后两个性质,这样可以把 $v_i$ 和 $w_i$ 加入前 $p$ 对点组成的环。

3. 对于 $i \in[s, t)$ , 连 $w_i \rightarrow v_0$. 因为最后一个性质,这样可以把 $w_i$ 加入环。

下面证明如何构造出一组满足条件的排列和对应的 $p$.

我们考虑收缩完 $\mathrm{SCC}$ 的图,建立一个超级源 $S$ ,向 $v_0, \ldots, v_{s-1}$ 连边. 建立一个超级汇 $T$ ,从 $w_0, \ldots, w_{t-1}$ 连边。

接下来,从 $S$ 到 $T$ 跑 Dinic 算法的\textbf{一次迭代},换言之,求一个阻塞流。

假设这个阻塞流大小是 $f$, 分别是从 $a_0 \rightsquigarrow b_0, \ldots, a_{f-1} \rightsquigarrow b_{f-1}$. 那么我们令 $p=f$. 那么第一个条件已经满足了。我们同时可以注意到,对于一条阻塞的边 $p \rightarrow q$. 那么存在一个阻塞的源点到 $q$, 从 $p$ 也可以到达一个阻塞的汇点。

对于一个非阻塞的源点,它肯定能到达一条阻塞的边,从而可以到达一个阻塞的汇点。非阻塞的汇点情况也是一样。于是就满足了后两个条件。

当然需要额外注意的是孤立的点,但是我们可以强行把这个点拆成两个点,转化为前面的问题。

\newpage

\subsection{2-SAT}

\subsubsection{建图}

$n$ 个节点,每个节点非 $0$ 即 $1$ ,$i_0$ 表示取 $0$ ,$i_1$ 表示取 $1$ 。

选了 $i_x$ 必须选 $j_y$ :$i_x\to j_y,j_{1-y}\to i_{1-x}$ 。

选了 $i_x$ 不能选 $j_y$ :$i_x\to j_{1-y},j_{y}\to i_{1-x}$ 。

$i$ 必须为 $x$ :$i_x\to i_{1-x}$ 。

\lstinputlisting{./code/graph/2-SAT.cpp}

\vspace{-0.4cm}

\subsubsection{前后缀优化建图}

\vspace{-0.2cm}

$n$ 个组,第 $i$ 组 $K_i$ 个点,每个组只能选一个。

对于单组,每个前缀新加一个点 $pre_i$ ,表示前 $i$ 个点是否选了:

1. $pre_{i-1}$ 选了必须选 $pre_{i}$ ,$pre_i$ 没选不能选 $pre_{i-1}$

2. $i$ 选了必须选 $pre_i$ ,$pre_i$ 没选不能选 $i$

3. $i$ 选了不能选 $pre_{i-1}$ ,$pre_{i-1}$ 选了不能选 $i$

\newpage

\subsection{圆方树(点双)}

\lstinputlisting{./code/graph/circlesquaretree.cpp}

\vspace{-0.6cm}

\subsection{带花树}

求解一般无向图的最大匹配,复杂度 $O(n^3+nm)$ ,稀疏图跑得飞快。

\lstinputlisting{./code/graph/blossomtree.cpp}

\newpage

\subsection{三元环}

无向图,度数小的节点连向度数大的节点(度数相同比编号),然后暴力枚举。

复杂度 $O(m\sqrt m)$ ,三元环最多 $O(m\sqrt m)$ 个。

\lstinputlisting{./code/graph/threecircle.cpp}

\subsection{四元环}

无向图,度数小的节点连向度数大的节点,枚举 $u$ 双向边连向的节点 $v$ ,枚举 $v$ 单向边连向的节点 $z$ ,如果 $u$ 度数小于 $z$ 则统计 $z$ 之前出现次数,并累加 $z$ 出现次数( $z$ 是度数最大点,避免计数重复)。

复杂度 $O(m\sqrt m)$ ,四元环可能很多。

\lstinputlisting{./code/graph/fourcircle.cpp}

\newpage


\subsection{一般图最小割}

$n$ 个点,$m$ 条边的无向图,问最小的边权和使得删掉这些边之后把图分成不连通的两部分。

复杂度 $O(n^3+nm)$ ,如果把找最大值改成堆就可以做到 $O(n^2+nm\log_2m)$ 。

\lstinputlisting{./code/graph/SW.cpp}

\newpage

\section{网络流}

\subsection{最大流}

\lstinputlisting{./code/graph/Dinic.cpp}

\vspace{-0.4cm}

\subsection{费用流}

\subsubsection{最小费用最大流}

\lstinputlisting{./code/graph/EKSpfa.cpp}

\subsubsection{最大费用可行流}

刷费用流的过程中,如果 $S$ 到 $T$ 增广路的费用 $\le 0$ ,即可停止网络流过程,得到最大费用可行流。

\subsubsection{负圈最小费用最大流}

边权 $w\ge 0$ 时,建 $x\to y$ 边权为 $w$ 的边。

边权 $w<0$ 时,建 $x\to y$ 边权为 $w$ ,上下界均为容量的边(实际表现为不用建),表示强制满流,并把费用计入答案。然后建 $y\to x$ 边权为 $-w$ 的边,用来退流。

然后刷 $S$ 到 $T$ 上下界最小费用最大流。

注意:这种费用流存在一种操作为把一个不经过 $S$ 和 $T$ 的环整体加流。

\lstinputlisting{./code/graph/circleEKSpfa.cpp}

\subsection{上下界网络流}

\subsubsection{无源汇上下界可行流}

先让所有边的流量都等于下界,然后再加上一些附加流,使得所有边流量守恒。

假设现在所有边的流量已经等于下界了。我们对于一个点 $i$ ,记录 $num[i]=$ $i$ 点流入量 $-$ $i$ 点流出量,那么如果:

1. $num[i]=0$ ,说明 $i$ 点是流量守恒的,不需要进行处理。

2. $num[i]>0$ ,说明 $i$ 点入 $>$ 出,连一条附加源 $SS\to i$ 的容量为 $num[i]$ 的边。

3. $num[i]<0$ ,说明 $i$ 点出 $>$ 入,连一条 $i\to$ 附加汇 $TT$ 的容量为 $-num[i]$ 的边。

如果存在一种可行流使得 $SS\to TT$ 满流,那么就说明原网络存在可行流。

原网络中每条边的流量是下界+流量(附加流)。

\subsubsection{有源汇上下界可行流}

连一条 $T\to S$ 的容量为 $∞$ 的边,变回无源汇上下界可行流。

可行流的流量是 $T\to S$ 这条边的流量。

\subsubsection{有源汇上下界最大流/最小流}

先求出一个有源汇上下界可行流,然后刷 $S\to T$ 的最大流。

在刷最大流之前把多余的边也就是与 $SS$ , $TT$ 相连的边以及 $T\to S$ 的边“干掉”(程序实现时使其满载即可)。

最小流即刷 $T\to S$ 的最大流。

上下界费用流把Dinic换成EK+Spfa即可。

\lstinputlisting{./code/graph/limitDinic.cpp}

\subsection{最小割}

\subsubsection{切糕模型}

有一块 $X\times Y\times Z$ 的切糕,每个点 $(x,y,z)$ 都有不和谐值 $v(x,y,z)$ 。现在要切这块切糕,为每个直线 $(x,y)$ 选出一个点 $z$ ,一种切法的不和谐值为选出点的不和谐值之和,为了平整还需要满足两条相邻的直线选出的点距离不超过 $D$ 。求最小不和谐值。

如果没有距离限制,我们可以把每个 $(x,y)$ 都从 $S$ 到 $T$ 串起来,像这样:$S \to 1 \to 2 \to 3 \to ... \to T$

将 $z-1\to z$ 的容量设置为 $v(x,y,z)$ ,然后求最小割就是最优解。

但是有限制,我们可以用容量为INF的边实现“屏蔽”效果,对于网络中的点 $(x,y,z)$ ,在 $(x,y,z-1)\to (x',y',z-1-D)$ 和 $(x',y',z+D)\to (x,y,z)$ 之间连容量为INF的边。

这样就会形成一个 $S\to (x,y,z-1)\to (x',y',z-1-D)\to (x',y',z+D)\to (x,y,z)\to T$ 的“通路”,从而阻止其他边被选。 

\subsubsection{占领模型}

$n\times m$ 的网格,有两种方法占领一个格子:1.花费 $c_{i,j}$ 。2.该格子上下左右的格子已经被占领。

占领一个格子之后有 $b_{i,j}$ 的收益,求收益减去花费的最大值。

先假装所有点都占领了,明显不合法,所以对一个点要么买下来要么把周围的买下来,或抛弃收益。

建边:把 $x$ 拆为 $x_{in}$ 和 $x_{out}$ ,连 $x_{in}\to x_{out}:b_x$ 。对于相邻的节点 $y$ :

$S\to x_{in}:c_x,x_{in}\to y_{in}:INF,x_{out}\to y_{out}:INF,y_{out}\to T:c_y$

为避免成环,将网格黑白染色。

\subsubsection{最大权闭合图}

给出一张带点权(有负权)的有向图,你可以选出一些点,如果这些点中任意一个点连接到的点均在选出的点中,则称这张图是闭合图,得到的价值是选出的点的点权之和,求最大权闭合图。

如果 $x$ 的点权 $w_x>0$ ,就建 $(S,x,w_x)$ 的边,否则建 $(x,T,-w_x)$ 的边,然后这张图中的边 $(u,v)$ 就建 $(u,v,+\infty)$ ,答案:
$$
(\sum_{w_x>0}w_x)-MinCut(S,T)
$$
经典模型:无向图,对于正权边 $(x,y)$ ,如果 $x$ 和 $y$(负权)选了,就可以获得这条边的贡献,求最大贡献。把边也当成点,选了 $(x,y)$ 就必须选 $x$ 和 $y$ ,求最大权闭合图即可。

\subsubsection{最大密度子图}

无向图,点权 $p_x$ ,边权 $w_e(w_e\ge 0)$ 。选出一些点 $V'$ ,求最大密度子图(不允许不选):
$$
{\sum_{x\in V'}p_x+\sum_{e\in E'}w_e\over|V'|},E'=\{(u,v)|u\in V',v\in V',(u,v)\in E\}
$$

01分数规划,二分答案 $mid$ ,得到:
$$
\sum_{e\in E'}w_e-\sum_{x\in V'}(mid-p_x)\ge 0\\
\sum_{e\in E'}W_e-\sum_{x\in V'}P_x\ge 0
$$
只要是形如 $W_e,P_x$ 的式子都可以用以下方法处理:

建 $S\to i:U$ ,$x\to y:W_e$ ,$i\to T:U+2P_i-d_i$ ,其中 $d_i$ 为与 $i$ 相连的所有边的 $W_e$ 之和,$U$ 是比所有 $d_i-2P_i$ 都大的数。

求 $S\to T$ 的最小割,上述式子的等价条件为 $Un-MinCut(S,T)>0$(不能 $=0$ ,否则说明没有选点)。

如果强制选 $i$ 点,只需要把 $S\to i$ 容量改成 $INF$ 。

求方案:从 $S$ 开始遍历非满载的边,能遍历到的点说明是被选的点。

\lstinputlisting{./code/graph/densitysubgraph.cpp}

\subsubsection{最小割选边}

给出一张 $n$ 个点 $m$ 条有向边的图,现在要求 $S,T$ 的最小割,问每一条边:1.有没有可能出现在最小割中。2.是否一定出现在最小割中。

先跑出随意一种最小割(最大流),然后在残量网络(没有满流的边)里缩点,结论:

1. 如果 $x\to y$ 没有满流,则一定不在最小割里。

2. $x\to y$ 可能出现:$SCC_x\not=SCC_y$ ;

3. $x\to y$ 一定出现:$SCC_S=SCC_x,SCC_y=SCC_T$ 。

\subsection{神秘的超快网络流/费用流模版}

\lstinputlisting{./code/graph/flow.cpp}

\subsection{Colin 的网络流/费用流模版}

\lstinputlisting{./code/graph/MF_gyx.cpp}

\lstinputlisting{./code/graph/MCMF_gyx.cpp}

