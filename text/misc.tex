\section{杂项}

\subsection{STL}

\subsubsection{bitset}

\lstinputlisting{./code/misc/bitset.cpp}

\subsubsection{配对堆(dijkstra)}

\lstinputlisting{./code/misc/pb_ds.cpp}

\newpage

\subsection{套路}

\subsubsection{等差数列乘积转阶乘}

$$
a(a+d)(a+2d)\cdots(a+nd)={a\over d}({a\over d}+1)({a\over d}+2)\cdots({a\over d}+n)\cdot d^{n+1}
$$

质模数小时可以转为阶乘。

\subsubsection{最小二乘法}

拟合后直线为 $y=kx+b$ :$k = \displaystyle\frac{n\sum_{i=1}^n x_iy_i - \big(\sum_{i=1}^n x_i \big)\big(\sum_{i=1}^n y_i\big)}{n\sum_{i=1}^n x_i^2 - \big(\sum_{i=1}^{n} x_i\big)^2}, b = \displaystyle\frac{\sum_{i=1}^{n} y_i - k\sum_{i=1}^n x_i}{n}$


\subsubsection{小范围不重复数点}

给 $n$ 组点 $(A,B,C)$ ,每个点覆盖 $\ge A,\ge B,\ge C$ 的部分,每组点覆盖的部分取并。

求 $(i,j,k)$ 被几组覆盖。

\lstinputlisting{./code/misc/countpoint.cpp}


\newpage

\subsubsection{gcd,and,or分块}

如果对每个右端点 $i$ ,记录 $suf_j$ 表示 $[j,i]$ 的 $\gcd$ ,将 $suf_j$ 相邻的相同值合并,最多只有 $\log$ 块,因为 $\gcd$ 变化时至少整除 $2$ 。

对于已经维护好的右端点 $i$ ,可以 $O(\log)$ 添加 $i+1$ 。

由于 $\text{and},\text{or}$ 也只会变化 $\log$ 次,因此也可以如上述处理。

\subsubsection{子集分块}

动态维护 $f_s=\sum_{t\subseteq S}a_t$ ,需要支持 $a_x$ 修改操作(超集同理)。

如果全集 $S$ 比较大,可以拆成两半,$t$ 是 $s$ 子集等价于两半都是 $s$ 子集。

定义 $f_{a,b}=\sum_{i=a,j\subseteq b}a_{i\cup j}$ ,从 $O(S)$ 修改 $O(1)$ 查询变为 $O(\sqrt S)$ 修改查询。

动态维护因子也可以这么处理,将 $n$ 拆为 $\prod_{k}p_k^{w_k}$ ,并分为 $\prod(w_i+1)$ 尽可能接近的两半,建出因子关系后同上述处理。


\subsubsection{二进制分组}

假设动态加入 $n$ 个可以合并的数据结构D,并且查询时不需要在合并后的数据结构中查询,可以采用二进制分组。

每次加入一个D时,先在右端加入一个大小为 $1$ 的组,然后查询 $Tail-1$ 和 $Tail$ 组大小是否一致,如果一致则暴力合并。

每个D只会被合并 $\log_2n$ 次。

\subsubsection{启发式分裂}

如果一个块 $[L,R]$ 能按规则分为两个块 $[L,m],[m+1,R]$ 且支持继续拆分,则可以从两侧同时往内部枚举,一侧遇到合法位置即停止枚举,这样复杂度是 $O(n\log_2n)$ 的,因为可以看作启发式合并的逆过程,每个点只会拆分 $O(\log_2n)$ 次。


\subsubsection{超级钢琴贪心模型}

选 $k$ 个长度在 $[L,R]$ 范围内的子段,求 $k$ 个子段的和的最大值。

类似选出前 $k$ 小的 $a_i+a_j$ ,定义 $Max(i,L,R)$ 表示 $i$ 作左端点,右端点在 $[L,R]$ 范围内的最大值,可以用ST表查询出 $Max(i,L,R)$ 中最优秀的右端点 $t$ ,然后将 $Max(i,L,R)$ 拆为 $Max(i,L,t-1),Max(i,t+1,R)$ ,用堆取 $k$ 次即可。


\subsubsection{平方计数}

求所有方案中 $\sum_{S} cnt(S,A)^2$ ,$cnt(S,A)$ 表示 $A$ 在方案 $S$ 中的出现次数。

可以转化为在 $S$ 中可重复的选择两次 $A$ 的方案数。

\subsubsection{循环矩阵乘法}

\vspace{-0.5cm}

$$
A_{i,j}=A_{0,(j-i)\bmod n}
$$
$$
(AB)_{0,k}=\sum_{i+j\equiv k\pmod n}A_{0,i}B_{0,j}
$$

将 $A_{i,j}$ 简记为 $A_i$ ,则 $AB$ 就是 $A$ 和 $B$ 的循环卷积。

\newpage

\subsubsection{树上解方程}

\vspace{-0.5cm}

$$
f(x)={1\over d_x}\sum_{(x,y)\in E}f(y)+w_x
$$

可以设 $f(x)=f(fa_x)+a_x$ ,如果 $w_x=1$ ,则:
$$
f(x)=f(fa_x)+sum_x
$$
其中 $sum_x$ 为 $x$ 子树中节点度数的和,即 $2si_x-1$ 。

\subsubsection{四连通解方程}

在 $x\in[-R,R],y\in [-R,R]$ 的四连通网格中解方程:
$$
f_{x,y}=p_{x,y,0}f_{x-1,y}+p_{x,y,1}f_{x,y-1}+p_{x,y,2}f_{x+1,y}+p_{x,y,3}f_{x,y+1}+a_{x,y}
$$
将可用格子从左上角到右下角编号,则每行方程只涉及 $2R$ 距离内的变元,因此每次高斯消元不需要 $O(R^2)$ ,只需要 $O(R)$ 。

\lstinputlisting{./code/misc/limitGaussian.cpp}


\subsubsection{去绝对值}

由于 $|x-y|=\max\{x-y,y-x\}$ ,如果要求最大值,则不合法的情况(负数)一定不会选。

枚举 $|x-y|$ 的正负情况后即可分别求 $\max\{\sum x\},\max\{\sum y\}$ 。

\newpage

\subsubsection{随机赋权判奇偶性}

利用随机赋权+异或,可以判整体奇偶性。

例:有 $n\times m$ 矩阵,每一行都有一个区间 $[L_i,R_i]$ 被覆盖,问多少个 $[A,B]$ 满足任意一行 $[A,B]$ 被覆盖的次数都是奇数/偶数。

给每行的 $[L_i,R_i]$ 随机一个权值,利用差分求出 $a_i$ 表示经过 $i$ 的线段的权值异或和,$d_i$ 表示 $L\le i$ 的权值异或和,记录 $sum_i$ 为 $a_i$ 的前缀异或和。

均出现偶数次:$sum_B\ \text{xor}\ sum_{A-1}=0$ 。

均出现奇数次:$sum_B\ \text{xor}\ sum_{A-1}=d_{B}\ \text{xor}\ d_{A}\ \text{xor}\ a_A$ ,即 $sum_B\ \text{xor}\ sum_{A}=d_{B}\ \text{xor}\ d_{A}$ 。


\subsubsection{FFT字符串匹配}

一般匹配:$\sum_{i=0}^{len-1}(A_i-B_i)^2=0$ 则匹配成功。

通配符匹配:如果串 $s$ 第 $i$ 位是通配符,则令 $S_i=0$ ,$\sum_{i=0}^{len-1}(A_i-B_i)^2A_iB_i=0$ 则匹配成功。

$K$ 次容错匹配:对于每个字符 $ch$ ,$S_i=[s_i=ch]$ ,$\sum_{i=0}^{len-1}A_iB_i$ 即 $ch$ 字符为 $A,B$ 提供的匹配数量。最后查询总匹配数量。

以上三个式子均可以对每个位置展开为卷积形式,然后用FFT快速求解。

\subsection{奇技淫巧}

\subsubsection{读入输出优化}

\lstinputlisting{./code/misc/readandwrite.cpp}

\subsubsection{模数非素数无法求逆元}

如果模数为非素数 $MOD$ ,但是整数运算最后要除以 $n$(且保证结果也是整数)。

可以运算时将模数改为 $MOD\cdot n$ ,就可以除以 $n$ 了。

\vspace{-0.3cm}

\subsubsection{模拟退火}

\vspace{-0.2cm}

\lstinputlisting{./code/misc/simulatedannealing.cpp}

\vspace{-0.4cm}

\subsubsection{中位数最小转权值最小}

$n$ 个元素,选出最少的个数满足条件 $A$ ,并且最少的前提下,权值中位数最小。

二分中位数,小于等于中位数的权值给 $3n-1$ ,大于中位数的权值给 $3n+1$ ,这样权值和最小就等价于选取最少(因为这样少选一个一定比多选一个权值和小)。

求出最小权值 $ans$ ,二分验证条件是 $3n-1$ 的个数大于等于 $3n+1$ 的个数。易知选取个数为 $cnt=\lfloor{ans+n\over 3n}\rfloor$ ,因此验证条件即 $ans\le 3n\cdot cnt$ 。

\vspace{-0.3cm}

\subsubsection{线性求逆元}

求出 $s_i=\prod_{j=1}^{i}a_j,sv_n=(s_n)^{-1},sv_{i}=sv_{i+1}a_{i+1}$ 。

则 $(a_i)^{-1}=s_{i-1}sv_i$ 。

\vspace{-0.3cm}

\subsubsection{线性快速幂}

要求 $a^{n}$ ,令 $S=\lceil \sqrt n\rceil$ ,预处理 $a^{iS}$ 和 $a^{j}$ ,则 $a^n=a^{iS}\cdot a^j$ ,预处理复杂度 $O(\sqrt n)$ 。

\vspace{-0.3cm}

\subsubsection{vscode配置}

快捷键:ctrl+shift+b 改为 ctrl+f9 。

配置:Auto Closing Brackets 和 Auto Closing Delete 改为 always ,Insert Spaces 取消勾选。

终端-配置默认生成任务,运行-添加配置。

\vspace{-0.3cm}

\subsubsection{Linux对拍}

调用方法:sh check.sh

\lstinputlisting{./code/misc/check.sh}

